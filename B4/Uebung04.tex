% Article template for Mathematics Magazine
% Revised 7/2002  Thanks for Greg St. George
\documentclass[12pt]{article}
\usepackage{amssymb}
\usepackage[ngerman]{babel}
\usepackage[utf8]{inputenc}
\usepackage{amsmath}
\renewcommand{\baselinestretch}{1.2}
%This is the command that spaces the manuscript for easy reading

\begin{document}
%\thispagestyle{empty}
\begin{center}
\Large
% TITLE GOES HERE
Logik und Komplexität  \textsc{ Übung 4 }
\end{center}

\begin{flushright}
Notizen zur dritten Übung\\
HU Berlin \\

\vspace{2 mm}

\end{flushright}

\subsubsection*{Aufgabe 1)}
\paragraph*{a)}
\[ 
\begin{array}{ccccccc}
  % \exists X ( &\exists x X(x) &\land \exists y \neg X(y) &\land \forall u \forall v ( &(X(u) \land \neg X(v) ) &\rightarrow (\neg E(u,v) \land &\neg E(v,u) ) ) )\\
  \exists x X(x) &\land \exists y \neg X(y) &\land \forall u \forall v ( (X(u) \land \neg X(v) ) &\rightarrow (\neg E(u,v) \land &\neg E(v,u) ) ) \\
  \bigvee_{a\in A} v_a &\land \bigvee_{a\in A}\neg v_a &\land \bigwedge_{a\in A} \bigwedge_{b\in A} ( v_a \land \neg v_b& \rightarrow (\neg 
  \begin{cases}
    1&(a,b) \in E \\
    0&(a,b) \notin E
  \end{cases}
  & \land \neg 
  \begin{cases}
    1&(b,a) \in E \\
    0&(b,a) \notin E
  \end{cases}
  ) )
\end{array} \\
\] 

\[ 
\begin{array}{rl}
   &     ( v_1 \lor v_2 \lor v_3) \\
  \land& (\neg v_1 \lor \neg v_2 \lor \neg v_3) \\
  \land& ( v_1 \land \neg v_1 \rightarrow \neg 0  \land \neg 0 ) \\
  \land& ( v_1 \land \neg v_2 \rightarrow \neg 1  \land \neg 0 ) \\
  \land& ( v_1 \land \neg v_3 \rightarrow \neg 0  \land \neg 0 ) \\
  \land& ( v_2 \land \neg v_1 \rightarrow \neg 0  \land \neg 0 ) \\
  \land& ( v_2 \land \neg v_2 \rightarrow \neg 0  \land \neg 0 ) \\
  \land& ( v_2 \land \neg v_3 \rightarrow \neg 0  \land \neg 0 ) \\
  \land& ( v_3 \land \neg v_1 \rightarrow \neg 0  \land \neg 0 ) \\
  \land& ( v_3 \land \neg v_2 \rightarrow \neg 0  \land \neg 0 ) \\
  \land& ( v_3 \land \neg v_3 \rightarrow \neg 1  \land \neg 1 ) \\
\end{array}
 \] 
 
 \paragraph{b)}
 Erfüllende Belegung ist $v_1, \neg v_3 $
 d.h.: Erfüllt für $X=\{1\}, X=\{1,2\}$
 
\subsubsection*{Aufgabe 2)}
Zeige: $Eval_{Fin_{<}}(\Phi)$ für jeden ESO-HORN-Satz $\Phi$ liegt in P.\\
D.h. es ex. ein det. Algorithmus der bei eingabe einer Endlichen Struktur $\mathfrak{A}$ in Polinomialzeit entscheidet, ob $\mathfrak{A} \models \Phi$ \\

Ein ESO-Horn Satz $\Phi$ besitzt die form:
\[  \exists X_1 ... \exists X_d \forall y_1 .. \forall y_{d'} \phi \] 

Wobei $\phi$ eine Konjunktion von ESO-Horn formeln ist:

\[ \phi := \bigwedge_{k\in \Gamma} \bigvee_{j\in K} \beta_{k,j} \] 

mit 
\[ \beta_{k,j} := 
  \begin{cases}
    \psi(y_1,...,y_{d'}) & \psi\in FO[\sigma]\\
    X_i(\bar y) & \bar y \in (\{y_1,...,y_d\}\cup\{c:\text{c ist eine Konstante in }\sigma\})^{ar(X_i)} \\
    \neg X_i(\bar y)& \bar y \in (\{y_1,...,y_d\}\cup\{c:\text{c ist eine Konstante in }\sigma\})^{ar(X_i)}
  \end{cases}
\] 

Aus dem Beweis von Theorem 2.24 konstruieren wir in \textbf{Polinomialzeit} bei eingabe von $\mathfrak{A}$ die Aussagenlogische Formel $\alpha_{\Phi,\mathfrak{A}}$. Diese hat die Form:

\[ \bigwedge_{a_1\in A}... \bigwedge_{a_d'\in A} \bigwedge_{k\in \Gamma} \bigvee_{j\in k} \beta'_{k,j} \] 

mit 

\[ \beta'_{k,j} := 
  \begin{cases}
    \psi(a_1,...,a_{d'}) & \psi\in FO[\sigma]\\
    v_{X_i,(\bar y)} & \bar y \in A^{ar(X_i)} \\
    \neg v_{X_i,(\bar y)} & \bar y \in A^{ar(X_i)} \\
  \end{cases}
\] 

$ \psi(a_1,...,a_{d'}) $ besitzt keine variablen und kann rekursiv in Polinomialzeit ausgewertet werden.
Somit ist $\Phi$ ein Aussagenlogischer Horn Satz ist, welcher mit dem Streichungsalgorithmus in Polinomialzeit gelöst werden kann.
$\square$

\subsubsection*{Aufgabe 3)}
 \paragraph{a)}
 $A \approx_2 B$ und $A \not\approx_3 B$ \\
 Gewinnstrategie Duplicator in 2 Runden: 'Wähle jeweils den entsprechenden Knoten aus der anderen Struktur. Sollte Spoiler aus B zwei der unteren äußeren 3 nehmen, wähle gegenüberliegende Knoten.'\\
 Gewinnstrategie Spoiler in 3 Runden: 'Wähle den linken, unteren und rechten Knoten in B. Denn es gibt keine drei knoten in A, so dass diese paarweise keine Kante haben.'
 
 \paragraph{b)}
 $\neg \Psi \equiv \exists x \exists y \exists z (x \neq y \land z\neq y \land x \neq z \land 
 \neg E(x,y) \land 
 \neg E(y,z) \land 
 \neg E(x,z))$ \\
 $ A \models \forall x \forall y \forall z (x = y \lor = y \lor x = z \lor 
  E(x,y) \lor 
  E(y,z) \lor 
  E(x,z))$ \\
 $ B \models \neg \Psi $

\subsubsection*{Aufgabe 4)}

\end{document}