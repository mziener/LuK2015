% Article template for Mathematics Magazine
% Revised 7/2002  Thanks for Greg St. George
\documentclass[12pt]{article}
\usepackage{amssymb}
\renewcommand{\baselinestretch}{1.2}
%This is the command that spaces the manuscript for easy reading

\begin{document}
%\thispagestyle{empty}
\begin{center}
\Large
% TITLE GOES HERE
A Sample Article for  \textsc{ Mathematics Magazine}
\end{center}

\begin{flushright}
Jack Q.~Firstauthor\footnote{Supported by the National Science
Foundation.}  \\
XXXX University \\
City, State 98765-4321\\
\verb+email@optional.edu+

\vspace{2 mm}

Jill P.~Secondauthor \\
Department of Physics
\footnote{Authors are in alphabetical order
unless there is an extraordinary reason to do otherwise.  Also,
the author address includes a department \emph{only} if
the department is \emph{not} mathematics. We use as few
footnotes as possible in the \textit{Magazine}.  This one, for instance,
contains information that really belongs in the body of the paper.
The previous footnote probably belongs among the Acknowledgments at the end.}\\
ZZZZ College \\
City, State 12345-6789
\end{flushright}

This document is meant to help you prepare an Article for
submission to \textsc{Mathematics Magazine}.  Of course,
editorial decisions depend entirely on
what you say and how you say it. Nonetheless, we will all save
time if you exercise some care in how you first present the paper.

Now that I have caught your attention with an interesting introductory paragraph,
here is what you will find:
specific information about the style of Articles in the \textsc{Magazine}
and a description of the \LaTeX\ code we prefer that you use
to prepare your manuscript.

Since this section is very clearly
an introduction, I thought that labeling it ``Introduction" would add
nothing.  Note that I am willing to use the first person in an Article
and you might be as well.  Another equally respectable choice is ``we,''
even when there is only one author; this can
create an author-reader partnership to
work through the mathematics together.
Whatever voice you choose, consistency is important.

You may be looking at this document in a variety of ways:  the .pdf or .ps files
are meant to be viewed on a screen or printed, while
the .tex file contains the codes used to create those
viewable versions via the program \LaTeX.  Even if you are a novice with \TeX,
there may be enough here to teach you what
you need to know.  And if you are an ace with \TeX,
we have a warning: please do not overload your
document with special kludges and tricks that will
only be removed later by our compositor.

This document is prepared with extremely simple \LaTeX\ formatting,
using the unadorned \verb+article+ template.
It is designed for simplicity and ease of handling---not
to imitate the \textsc{Magazine}'s final, typeset style in every
detail.    For authors less familiar with \LaTeX, we offer a
brief lesson, showing how certain common elements of mathematical
style are typeset using this program.  For hardcore technical
specifications, please see the Electronic Publication Guidelines~\cite{MAA}.

\subsection*{Notes on writing an Article}
Articles in the \textsc{Magazine} tend to be longer and
more substantial than Notes,
offering a broad overview of some field
or making new connections.
Being longer, they often
benefit from more sectioning. We use
the \verb+\subsection*+ command to create titles
for these sections, which are not usually numbered
(the \verb+*+ in \verb+\subsection*+ accomplishes this).

To judge the length of your piece,  consider that
this document prints to six pages with the current code, but would run
about four pages in the \textsc{Magazine}.  The current settings produce
a document that is generously spaced in consideration of
referees' eyesight.

Few pieces of mathematical writing are entirely self-contained,
although we try to make Articles reasonably so.    Consider
providing a section of background material that our
more knowledgeable readers can skip. Define
enough terms to enable an eager undergraduate student
to read your piece without having to consult
too many references.

For readers intrigued by your exposition,
you should provide friendly references.
Bibliographies may contain suggested reading along with sources actually referenced.  In all cases,
cite sources that are currently and readily available.

\LaTeX\ has a way to keep track of references automatically,
which is illustrated in the code that ends this file.  To refer
to Halmos~\cite{Halmos}, you use a codename that you have created
as a mnemonic, often the author's last name.
 \LaTeX\
 keeps track, numbering the references in
the order they appear in your list.  If you add a reference
(positioning it correctly in the list) the numbers will be adjusted accordingly.

Please follow our bibliographic format carefully, based
on the examples below.  Entries may appear either in alphabetical
order or in order of citation (but choose one order and
stick to it).   Journal titles are abbreviated
as in \emph{Mathematical Reviews},
for instance, \textit{Amer. Math. Monthly}; volume numbers
of journals are set in \textbf{bold}.  Authors names are
not inverted: Frank A. Farris, not Farris, Frank A.~\cite{Farris}.
The abbreviation pp. is used for books, but not journal
articles.  Note the slightly different style for
citing articles in the \textsc{Magazine}.


\paragraph*{How to do things in \LaTeX}

Roman letters used as variables will be correctly
italicized if enclosed with \$s in your code, as in ``functions $f$, $g$, and $h$.''
This makes for typing lots of \$s when writing in \TeX.
Other popular fonts are $\mathcal A$, for sets and the like, and
${\mathbb Z}$ for the integers, etc.

This last symbol, the ``blackboard'' ${\mathbb Z}$, actually
is not part of basic \LaTeX.  If you look in the \textit{preamble}
of this document, the part before the \verb+\begin{document}+
command, you will see the instruction \verb+\usepackage{amssymb}+ .
This enables you to use the blackboard font, as well as certain special symbols:
$$\lceil \  , \rceil \  , \lfloor \  , \rfloor  \ , \mathrm {\ and  \ so \  on.}$$
\noindent
If you do not have this package, you are welcome to mark these
symbols in by hand. While we are talking about packages,
please do not use any package that redefines major environments,
such as the theorem environment.

\LaTeX\ is able to number theorems automatically,
using what is called a theorem
\textit{environment}.  This is usually overkill for pieces in the \textsc{Magazine}.
The following example shows a simple method for displaying
theorems; the theorem need not even be numbered unless
you refer to it by number later.

\textsc{Theorem 1.}

Let $a$ be any real number.   Then $a^2 > -1$.


\textit{Proof}.  The result follows from well-known
properties of flabby sheaf cohomology over algebraically closed
fields. This parody of a proof, the likes of which you would not see
in the \textsc{Magazine}, ends here, but you don't need to insert
an end-of-proof marker.  You could put a comment in the file to mark
the end of the proof.
%%End of Proof


A remarkable result that has been the target of many proofs in the
\textsc{Magazine} is the Pythagorean theorem.    If $a$, $b$, and
$c$ are the sides of a right triangle, then
\begin{equation}
a^2+b^2=c^2 .  \label{Pythagoras}
\end{equation}
% i modified the next paragraph --- \eqno is obsolete.
The equation above is called a \textit{displayed equation}.  The
reference number was added using the equation environment
(enclosing the code for the equation between
\verb+\begin{equation}+ and \verb+{\end{equation}+). You should
give numbers only to those equations that you cite by number
later; to refer to his equation without having to remember which
number it had, we gave it a descriptive label, Pythagoras, whose
use is shown in the code below.  The sentence ended with the
equation, so we used a period.

It can be shown from equation \ref{Pythagoras}, by means of a
routine calculation, that $b^2+a^2=c^2$.   Indeed, many related
equations can be derived, such as these:
\begin{eqnarray*}
a^2-c^2 &=& -b^2   \\
b^2-c^2 &=& -a^2 .
\end{eqnarray*}
This is an example of a \LaTeX\ environment that you may find
useful; it aligns the equations on the equals sign. The asterisk
in the code \verb+\begin{eqnarray*}+ suppresses numbering.To
display a single equation without numbering it, enclose the code
in a pair of double \$s, as shown below.

Another useful environment is \verb+tabular+. Note that
environments must have \verb+begin+ and \verb+end+ markers.  The
code that makes the brace shows how \LaTeX\ uses the commands
\verb+\left+ and \verb+\right+ to
resize delimiters automatically. This also demonstrates the \verb+\center+
environment.
\vskip 2mm
\begin{center}
$$\left (
\begin{tabular}{lcr}
This text & is arranged& in a table\\
with an ampersand \& & to delimit& columns\\
and double & backslashes& to end rows.
\end{tabular}
\right )$$
\vspace{3mm}
\end{center}
The main point of your interesting Article
might be illustrated by something like \textsc{Figure 1},
which I did not include in this template, partly because it's fictional,
but also because I would have had to provide an additional
electronic file for you to download (presumably
in \textit{encapsulated PostScript} format, eps).
Figures should have brief explanatory captions,  like
\verb+Figure 1 Worth 1000 words+, without
a period at the end.
Figures may be included in the printed output, using
a package such as \verb+epsfig+.
The simplest alternative is to put the figures
at the end, and note:   \textbf{FIGURE 1 GOES NEAR HERE}.

In the previous paragraph, a portion of the
text hangs out into the right margin.  \TeX\ did not
know how to hyphenate the verbose text string.  In
preparing your manuscript, you do not need to worry about
things like this.  The line breaks will all change later anyway.
There is much to say about producing figures that will
look good in print.  Before you invest a great deal of time
creating figures, please read the detailed Electronic
Production Guidelines~\cite{MAA}.  Here, suffice it to say
that we strongly prefer PostScript formats for figures.
Programs such as \textit{Maple} and \textit{Mathematica}
give you the option of saving pictures this way.  \textit{Geometer's Sketchpad} does not,
but our compositor knows
how to handle this type of file.

\paragraph*{Conclusion}
Now replace all the text in this file with a crackling exposition of
a sweeping mathematical panorama, \TeX\ it up, and send three hard copies to:
\vskip 3mm

\noindent Frank A. Farris, Editor, \textsc{Mathematics Magazine}, Santa Clara University,\\
\noindent 500 El Camino Real, Santa Clara, CA 95053-0373
\vskip 3mm

Electronic submission is possible in limited circumstances;
ultimately, we need two hard copies to send to referees and one to
retain.  To request that we do this printing for you, please inquire
at \verb+mathmag@scu.edu+ or 408-554-4122.

\paragraph*{Acknowledgment}
We thank our spouses, the anonymous referees, granting agencies, and our moms
for everything they've done for us.    If the editor
helped, that's fine, but we don't thank him here
since he's only doing his job.


\begin{thebibliography}{20}

\bibitem{Boas}  R.P. Boas, Can we make mathematics intelligible? \textit{Amer. Math.
Monthly} \textbf{88} (1981), 727--731.

\bibitem{Farris}  Frank Farris, \textit{A \textsc{Mathematics Magazine} Retrospective}, this \textsc{Magazine}, \textbf{79} (2006), 1-88.

\bibitem{Halmos} Paul Halmos, How to write mathematics,
\textit{Enseign.  Math.} \textbf{16} (1970), 123--152.  Reprinted in
Halmos, \textit{Selecta, expository writings}, Vol.  2, Springer, New
York, 1983, 157--186.

\bibitem{Hwang} Andrew Hwang, Writing in the age of Latex, \textit{AMS
Notices} \textbf{42} (1995), 878--882.

\bibitem{Knuth}  D.E. Knuth, T. Larrabee, and P.M. Roberts,
\textit{Mathematical Writing}, MAA Notes \#14, 1989.

\bibitem{Krantz}   Steven G. Krantz, \textit{A Primer of Mathematical Writing},
American Mathematical Society, 1997.

\bibitem{MAA} Mathematical Association of America, \textit{Electronic
Production Guidelines}, http://www.maa.org/pubs/bev.html .

\bibitem{Mermin} N. David Mermin, \textit{Boojums All the Way
Through}, Cambridge Univ.  Pr., Cambridge, UK, 1990.

\end{thebibliography}



\end{document}
